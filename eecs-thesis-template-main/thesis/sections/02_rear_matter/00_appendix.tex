\begin{appendices}
\cleardoublepage

\chapter{System Intialization and Testing}

The below sections detail the initial setup and testing of the Raspberry Pi 5 and different SDR hardware modules.

\section{Python Code for LimeSDR Signal Capture \label{app:limeSDRscript}}
\begin{lstlisting}[language=Python, caption={Python code for capturing DAB signal using LimeSDR}, label={lst:limeSDR_code}]
import numpy as np
import os
import scipy
from matplotlib import pyplot as plt
import SoapySDR
from SoapySDR import SOAPY_SDR_RX, SOAPY_SDR_CS16

########################################################################################
# Settings
########################################################################################
# Data transfer settings
rx_chan = 0               # RX1 = 0, RX2 = 1 - Using the RX1 Wideband
N = 2048000                  # Number of complex samples per transfer
fs = 2.048e6               # Radio sample Rate
freq = 202.928e6               # LO tuning frequency in Hz - DAB 9A
use_agc = True           # Use or don't use the AGC
timeout_us = int(5e6)

# Recording Settings
cplx_samples_per_file = 2048000  # Complex samples per file
nfiles = 1              # Number of files to record
rec_dir = '/Users/flynnmkelly/Desktop/Thesis/PassiveRadarThesis/LimeSDR'  # Location of drive for recording
file_prefix = 'TestFile'   # File prefix for each file

########################################################################################
# Receive Signal
########################################################################################
# File calculations and checks
cplx_samples_per_file = N   # Use entire buffer size for one file
real_samples_per_file = 2 * cplx_samples_per_file

# Initialize the LimeSDR receiver
sdr = SoapySDR.Device(dict(driver="lime"))
print(sdr.listSensors())  # Print sensor information

sdr.setSampleRate(SOAPY_SDR_RX, 0, fs)          # Set sample rate
sdr.setGainMode(SOAPY_SDR_RX, 0, use_agc)       # Set gain mode
sdr.setFrequency(SOAPY_SDR_RX, 0, freq)         # Tune to DAB frequency

print('Configuration complete')

# Create data buffer and start streaming
rx_buff = np.empty(2 * N, np.int16)  # Buffer for data
rx_stream = sdr.setupStream(SOAPY_SDR_RX, SOAPY_SDR_CS16, [rx_chan]) # Setup data stream
sdr.activateStream(rx_stream)  # Start streaming

# Record the data
sr = sdr.readStream(rx_stream, [rx_buff], N, timeoutUs=timeout_us)
rc = sr.ret
assert rc == N, 'Error Reading Samples from Device (error code = %d)!' % rc

# Save data to file
file_name = os.path.join(rec_dir, '{}.bin'.format(file_prefix))
rx_buff.tofile(file_name)

# Stop streaming and close connection
sdr.deactivateStream(rx_stream)
sdr.closeStream(rx_stream)
\end{lstlisting}

% ------------------------------

\section{Code to Plot PSD for Testing LimeSDR}

\begin{lstlisting}[language=Python, caption={Main Code to Read Data and Plot PSD}, label={lst:main_code}]
# Main Code to Read Data and Plot PSD
file_name = 'TestFile.bin'  # Specify your file name here
fs = 2.048e6  # Sampling frequency
center_freq = 202.928e6  # Center frequency

# Read complex data from the file
complex_data = read_complex_int16(file_name)

# Compute the power spectral density
f, Pxx = welch(complex_data, fs, nperseg=2048)

# Shift the frequency axis
f_shifted = f + center_freq

# Plot PSD
plt.figure(figsize=(12, 6))
plt.semilogy(f_shifted, Pxx)
plt.title('Power Spectral Density of the DAB Signal')
plt.xlabel('Frequency (Hz)')
plt.ylabel('Power/Frequency (dB/Hz)')
plt.grid(True)

# Set x-axis limits to show the full spectrum around the center frequency
plt.xlim(center_freq - fs/2, center_freq + fs/2)

# Format x-axis ticks to show MHz instead of Hz
plt.gca().xaxis.set_major_formatter(plt.FuncFormatter(lambda x, p: f"{x/1e6:.3f}"))
plt.xlabel('Frequency (MHz)')

plt.show()
\end{lstlisting}

\end{appendices}