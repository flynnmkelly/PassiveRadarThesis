\begin{appendices}
\cleardoublepage

\chapter{Utilised Code Listings}

The code used throughout this 

\chapter{System Intialization and Testing}

The below sections detail the initial setup and testing of the Raspberry Pi 5 and different SDR hardware modules.

\section{SoapySDRUtil LimeSDR Probe Output}
\begin{lstlisting}[language=bash, caption={SoapySDRUtil Probe Output for LimeSDR}, label={lst: soapyProbe}]

[INFO] Make connection: 'LimeSDR-USB [USB 3.0] 90706024F3821'
libusb: warning [darwin_transfer_status] transfer error: timed out
[INFO] Reference clock 30.72 MHz
[INFO] Device name: LimeSDR-USB
[INFO] Reference: 30.72 MHz
[INFO] LMS7002M register cache: Disabled

----------------------------------------------------
-- Device identification
----------------------------------------------------
  driver=FX3
  hardware=LimeSDR-USB
  boardSerialNumber=0x90706024f3821
  firmwareVersion=4
  gatewareVersion=2.23
  hardwareVersion=4
  protocolVersion=1

----------------------------------------------------
-- Peripheral summary
----------------------------------------------------
  Channels: 2 Rx, 2 Tx
  Timestamps: YES
  Clock sources: internal, external
  Sensors: clock_locked, lms7_temp
     * clock_locked (Clock Locked): true
        CGEN clock is locked, good VCO selection.
     * lms7_temp (LMS7 Temperature): 41.495247 C
        The temperature of the LMS7002M in degrees C.
  Registers: BBIC, RFIC0
  Other Settings:
     * SAVE_CONFIG - Save LMS settings to file
       [key=SAVE_CONFIG, type=string]
     * LOAD_CONFIG - Load LMS settings from file
       [key=LOAD_CONFIG, type=string]
     * OVERSAMPLING - oversampling ratio (0 - auto)
       [key=OVERSAMPLING, type=int, options=(0, 1, 2, 4, 8, 16, 32)]
  GPIOs: MAIN

----------------------------------------------------
-- RX Channel 0
----------------------------------------------------
  Full-duplex: YES
  Supports AGC: NO
  Stream formats: CF32, CS12, CS16
  Native format: CS16 [full-scale=32767]
  Stream args:
     * Buffer Length - The buffer transfer size over the link.
       [key=bufferLength, units=samples, default=0, type=int]
     * Latency - Latency vs. performance
       [key=latency, default=0.5, type=float]
     * Link Format - The format of the samples over the link.
       [key=linkFormat, default=CS16, type=string, options=(CS16, CS12)]
     * Skip Calibration - Skip automatic activation calibration.
       [key=skipCal, default=false, type=bool]
     * align phase - Attempt to align phase of Rx channels.
       [key=alignPhase, default=false, type=bool]
  Antennas: NONE, LNAH, LNAL, LNAW, LB1, LB2
  Corrections: DC removal, DC offset, IQ balance
  Full gain range: [-12, 61] dB
    TIA gain range: [0, 12] dB
    LNA gain range: [0, 30] dB
    PGA gain range: [-12, 19] dB
  Full freq range: [0, 3800] MHz
    RF freq range: [30, 3800] MHz
    BB freq range: [-10, 10] MHz
  Tune args:
     * LO Offset - Tune the LO with an offset and compensate with the baseband CORDIC.
       [key=OFFSET, units=Hz, default=0.0, type=float, range=[-1e+07, 1e+07]]
     * BB - Specify a specific value for this component or IGNORE to skip tuning it.
       [key=BB, units=Hz, default=DEFAULT, type=float, range=[-1e+07, 1e+07], options=(DEFAULT, IGNORE)]
  Sample rates: [0.1, 61.44] MSps
  Filter bandwidths: [1.4001, 130] MHz
  Sensors: lo_locked
     * lo_locked (LO Locked): false
        LO synthesizer is locked, good VCO selection.
  Other Settings:
     * TSP_CONST - Digital DC test signal level in LMS7002M TSP chain.
       [key=TSP_CONST, default=16383, type=int, range=[0, 32767]]
     * CALIBRATE -  DC/IQ calibration bandwidth
       [key=CALIBRATE, type=float, range=[2.5e+06, 1.2e+08]]
     * ENABLE_GFIR_LPF - LPF bandwidth (must be set after sample rate)
       [key=ENABLE_GFIR_LPF, type=float]
     * TSG_NCO - Enable NCO test signal
       [key=TSG_NCO, default=4, type=int, options=(-1, 4, 8)]

----------------------------------------------------
-- RX Channel 1
----------------------------------------------------
  Full-duplex: YES
  Supports AGC: NO
  Stream formats: CF32, CS12, CS16
  Native format: CS16 [full-scale=32767]
  Stream args:
     * Buffer Length - The buffer transfer size over the link.
       [key=bufferLength, units=samples, default=0, type=int]
     * Latency - Latency vs. performance
       [key=latency, default=0.5, type=float]
     * Link Format - The format of the samples over the link.
       [key=linkFormat, default=CS16, type=string, options=(CS16, CS12)]
     * Skip Calibration - Skip automatic activation calibration.
       [key=skipCal, default=false, type=bool]
     * align phase - Attempt to align phase of Rx channels.
       [key=alignPhase, default=false, type=bool]
  Antennas: NONE, LNAH, LNAL, LNAW, LB1, LB2
  Corrections: DC removal, DC offset, IQ balance
  Full gain range: [-12, 61] dB
    TIA gain range: [0, 12] dB
    LNA gain range: [0, 30] dB
    PGA gain range: [-12, 19] dB
  Full freq range: [0, 3800] MHz
    RF freq range: [30, 3800] MHz
    BB freq range: [-10, 10] MHz
  Tune args:
     * LO Offset - Tune the LO with an offset and compensate with the baseband CORDIC.
       [key=OFFSET, units=Hz, default=0.0, type=float, range=[-1e+07, 1e+07]]
     * BB - Specify a specific value for this component or IGNORE to skip tuning it.
       [key=BB, units=Hz, default=DEFAULT, type=float, range=[-1e+07, 1e+07], options=(DEFAULT, IGNORE)]
  Sample rates: [0.1, 61.44] MSps
  Filter bandwidths: [1.4001, 130] MHz
  Sensors: lo_locked
     * lo_locked (LO Locked): false
        LO synthesizer is locked, good VCO selection.
  Other Settings:
     * TSP_CONST - Digital DC test signal level in LMS7002M TSP chain.
       [key=TSP_CONST, default=16383, type=int, range=[0, 32767]]
     * CALIBRATE -  DC/IQ calibration bandwidth
       [key=CALIBRATE, type=float, range=[2.5e+06, 1.2e+08]]
     * ENABLE_GFIR_LPF - LPF bandwidth (must be set after sample rate)
       [key=ENABLE_GFIR_LPF, type=float]
     * TSG_NCO - Enable NCO test signal
       [key=TSG_NCO, default=4, type=int, options=(-1, 4, 8)]

----------------------------------------------------
-- TX Channel 0
----------------------------------------------------
  Full-duplex: YES
  Supports AGC: NO
  Stream formats: CF32, CS12, CS16
  Native format: CS16 [full-scale=32767]
  Stream args:
     * Buffer Length - The buffer transfer size over the link.
       [key=bufferLength, units=samples, default=0, type=int]
     * Latency - Latency vs. performance
       [key=latency, default=0.5, type=float]
     * Link Format - The format of the samples over the link.
       [key=linkFormat, default=CS16, type=string, options=(CS16, CS12)]
     * Skip Calibration - Skip automatic activation calibration.
       [key=skipCal, default=false, type=bool]
     * align phase - Attempt to align phase of Rx channels.
       [key=alignPhase, default=false, type=bool]
  Antennas: NONE, BAND1, BAND2
  Corrections: DC offset, IQ balance
  Full gain range: [-12, 64] dB
    PAD gain range: [0, 52] dB
    IAMP gain range: [-12, 12] dB
  Full freq range: [0, 3800] MHz
    RF freq range: [30, 3800] MHz
    BB freq range: [-10, 10] MHz
  Tune args:
     * LO Offset - Tune the LO with an offset and compensate with the baseband CORDIC.
       [key=OFFSET, units=Hz, default=0.0, type=float, range=[-1e+07, 1e+07]]
     * BB - Specify a specific value for this component or IGNORE to skip tuning it.
       [key=BB, units=Hz, default=DEFAULT, type=float, range=[-1e+07, 1e+07], options=(DEFAULT, IGNORE)]
  Sample rates: [0.1, 61.44] MSps
  Filter bandwidths: [5, 40], [50, 130] MHz
  Sensors: lo_locked
     * lo_locked (LO Locked): true
        LO synthesizer is locked, good VCO selection.
  Other Settings:
     * TSP_CONST - Digital DC test signal level in LMS7002M TSP chain.
       [key=TSP_CONST, default=16383, type=int, range=[0, 32767]]
     * CALIBRATE -  DC/IQ calibration bandwidth
       [key=CALIBRATE, type=float, range=[2.5e+06, 1.2e+08]]
     * ENABLE_GFIR_LPF - LPF bandwidth (must be set after sample rate)
       [key=ENABLE_GFIR_LPF, type=float]
     * TSG_NCO - Enable NCO test signal
       [key=TSG_NCO, default=4, type=int, options=(-1, 4, 8)]

----------------------------------------------------
-- TX Channel 1
----------------------------------------------------
  Full-duplex: YES
  Supports AGC: NO
  Stream formats: CF32, CS12, CS16
  Native format: CS16 [full-scale=32767]
  Stream args:
     * Buffer Length - The buffer transfer size over the link.
       [key=bufferLength, units=samples, default=0, type=int]
     * Latency - Latency vs. performance
       [key=latency, default=0.5, type=float]
     * Link Format - The format of the samples over the link.
       [key=linkFormat, default=CS16, type=string, options=(CS16, CS12)]
     * Skip Calibration - Skip automatic activation calibration.
       [key=skipCal, default=false, type=bool]
     * align phase - Attempt to align phase of Rx channels.
       [key=alignPhase, default=false, type=bool]
  Antennas: NONE, BAND1, BAND2
  Corrections: DC offset, IQ balance
  Full gain range: [-12, 64] dB
    PAD gain range: [0, 52] dB
    IAMP gain range: [-12, 12] dB
  Full freq range: [0, 3800] MHz
    RF freq range: [30, 3800] MHz
    BB freq range: [-10, 10] MHz
  Tune args:
     * LO Offset - Tune the LO with an offset and compensate with the baseband CORDIC.
       [key=OFFSET, units=Hz, default=0.0, type=float, range=[-1e+07, 1e+07]]
     * BB - Specify a specific value for this component or IGNORE to skip tuning it.
       [key=BB, units=Hz, default=DEFAULT, type=float, range=[-1e+07, 1e+07], options=(DEFAULT, IGNORE)]
  Sample rates: [0.1, 61.44] MSps
  Filter bandwidths: [5, 40], [50, 130] MHz
  Sensors: lo_locked
     * lo_locked (LO Locked): true
        LO synthesizer is locked, good VCO selection.
  Other Settings:
     * TSP_CONST - Digital DC test signal level in LMS7002M TSP chain.
       [key=TSP_CONST, default=16383, type=int, range=[0, 32767]]
     * CALIBRATE -  DC/IQ calibration bandwidth
       [key=CALIBRATE, type=float, range=[2.5e+06, 1.2e+08]]
     * ENABLE_GFIR_LPF - LPF bandwidth (must be set after sample rate)
       [key=ENABLE_GFIR_LPF, type=float]
     * TSG_NCO - Enable NCO test signal
       [key=TSG_NCO, default=4, type=int, options=(-1, 4, 8)]


\end{lstlisting}

\section{SoapySDRUtil RTL-SDR Probe Output}
\begin{lstlisting}[language=bash, caption={SoapySDRUtil Probe Output for RTL-SDR}, label={lst: soapyProbeRTL}]

    Found Rafael Micro R828D tuner
    [INFO] Opening Generic RTL2832U OEM :: 00000001...
    Found Rafael Micro R828D tuner
    
    ----------------------------------------------------
    -- Device identification
    ----------------------------------------------------
      driver=RTLSDR
      hardware=R828D
      index=0
      origin=https://github.com/pothosware/SoapyRTLSDR
    
    ----------------------------------------------------
    -- Peripheral summary
    ----------------------------------------------------
      Channels: 1 Rx, 0 Tx
      Timestamps: YES
      Time sources: sw_ticks
      Other Settings:
         * Direct Sampling - RTL-SDR Direct Sampling Mode
           [key=direct_samp, default=0, type=string, options=(0, 1, 2)]
         * Offset Tune - RTL-SDR Offset Tuning Mode
           [key=offset_tune, default=false, type=bool]
         * I/Q Swap - RTL-SDR I/Q Swap Mode
           [key=iq_swap, default=false, type=bool]
         * Digital AGC - RTL-SDR digital AGC Mode
           [key=digital_agc, default=false, type=bool]
         * Bias Tee - RTL-SDR Blog V.3 Bias-Tee Mode
           [key=biastee, default=false, type=bool]
    
    ----------------------------------------------------
    -- RX Channel 0
    ----------------------------------------------------
      Full-duplex: NO
      Supports AGC: YES
      Stream formats: CS8, CS16, CF32
      Native format: CS8 [full-scale=128]
      Stream args:
         * Buffer Size - Number of bytes per buffer, multiples of 512 only.
           [key=bufflen, units=bytes, default=262144, type=int]
         * Ring buffers - Number of buffers in the ring.
           [key=buffers, units=buffers, default=15, type=int]
         * Async buffers - Number of async usb buffers (advanced).
           [key=asyncBuffs, units=buffers, default=0, type=int]
      Antennas: RX
      Full gain range: [0, 49.6] dB
        TUNER gain range: [0, 49.6] dB
      Full freq range: [23.999, 1764] MHz
        RF freq range: [24, 1764] MHz
        CORR freq range: [-0.001, 0.001] MHz
      Sample rates: [0.225001, 0.3], [0.900001, 3.2] MSps
      Filter bandwidths: [0, 8] MHz
    

\end{lstlisting}

\section{Crontab Configuration}
\begin{lstlisting}[language=bash, caption={Crontab Configuration for Raspberry Pi}, label={lst:crontab}]
flynnmkelly@raspberrypi:~ $ sudo crontab -l
# Edit this file to introduce tasks to be run by cron.
# 
# Each task to run has to be defined through a single line
# indicating with different fields when the task will be run
# and what command to run for the task
# 
# To define the time you can provide concrete values for
# minute (m), hour (h), day of month (dom), month (mon),
# and day of week (dow) or use '*' in these fields (for 'any').
# 
# Notice that tasks will be started based on the cron's system
# daemon's notion of time and timezones.
# 
# Output of the crontab jobs (including errors) is sent through
# email to the user the crontab file belongs to (unless redirected).
# 
# For example, you can run a backup of all your user accounts
# at 5 a.m every week with:
# 0 5 * * 1 tar -zcf /var/backups/home.tgz /home/
# 
# For more information see the manual pages of crontab(5) and cron(8)
# 
# m h  dom mon dow   command
@reboot sh /home/flynnmkelly/Desktop/Testbed/launcher.sh >/home/flynnmkelly/logs/cronlog 2>&1
\end{lstlisting}




\end{appendices}